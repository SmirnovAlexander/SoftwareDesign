\documentclass[a4paper,10pt]{article}

% Encoding.
\usepackage{geometry}
\usepackage[T2A]{fontenc}
\usepackage[utf8]{inputenc}
\usepackage[english,russian]{babel}

% Code insertion.
\usepackage[outputdir=temp]{minted}

% Math functions.
\usepackage{amsmath}

% Image insertion.
\usepackage{svg}

% No line breaks.
\usepackage[none]{hyphenat}

% \usepackage{indentfirst}
\setlength\parindent{0pt}

\title{Задание 2: проектирование Roguelike}
\author{
	Жилкин Феодор\\
	\and
	Смирнов Александр
}
\date{\today}

\begin{document}


\maketitle

\section*{Схема работы}

\subsection*{Начало игры}

Игра начинается с генерации игрового мира (т.е. карты). Пользователь может выбрать загрузку карты из файла или же рандомную генерацию карты. 

\subsection*{Персонажи}

Мы играем за Героя (который наследуется от родительского класса Персонаж). Также в игре представлены разного рода мобы, которые также являются персонажами. 
Всех персонажей объединяют следующие \textbf{особенности} (поля, свойства): 

\begin{itemize}
    \item \textbf{HP} -- уровень здоровья;
    \item \textbf{Damage} -- количество наносимого урона;
    \item \textbf{Resist} -- количество защиты;
    \item \textbf{Equipment} -- экипировка (список из предметов, которые находятся на персонаже);
    \item \textbf{Level} -- текущий уровень персонажа.
\end{itemize}


и следующие \textbf{действия} (поведение):


\begin{itemize}
    \item \textbf{Attack} -- возможность атаковать другого персонажа;
    \item \textbf{Move} -- возможность перемещаться по карте.
\end{itemize}

\subsubsection*{Свойства}

Свойство \textbf{Damage} персонажа рассчитывается по 3-м параметрам: 

\begin{itemize}
    \item количество очков атаки оружия;
    \item количество очков атаки боеприпасов (для оружия дальнего боя);
    \item количество очков атаки самого персонажа.
\end{itemize}

Свойство \textbf{Resist} персонажа рассчитывается по 2-м параметрам:

\begin{itemize}
    \item количество очков защиты всей брони на персонаже (Armor);
    \item количество очков защиты самого персонажа.
\end{itemize}

\subsubsection*{Герой}

Герой (Hero), за которого мы играем также имеет \textbf{инвентарь} (Inventory) -- список вещей, которые герой носит вместе с собой. 

\subsubsection*{Мобы}

Мобы (Mob) являются теми же персонажами, что и наш, герой, но они выступают в качестве главных антагонистов в игре (именно против них мы и будем отважно сражаться в темных залах подземелья, не боясь потерять свою честь или даже жизнь\ldots).

Помимо этого у мобов нет инвентаря. У мобов есть 3 состояния и 3 поведения описывающие эти состояния:


\begin{itemize}
    \item \textbf{Агрессивный режим} (Rage Mode) и агрессивная стратегия поведения;
    \item \textbf{Пугливый режим} (Coward Mode) и пугливая (отступающая) стратегия поведения;
    \item \textbf{Пассивный режим} (Passive Mode) и пассивная стратегия поведения.
\end{itemize}

В зависимости от количества здоровья мобы могут переходить в разные режимы и преследовать соответствующую модель поведения (стратегию). Например, если у изначально агрессивно настроенного моба малое количество здоровья, то он переходит в трусливый режим и бросается в бегство. Снова в нападение жалкий трус пойдет только тогда, когда уровень его здоровья повысится. 

За такую смену состояния (режима) и поведения (стратегии) отвечают паттерны «Состояние» и «Стратегия» соответственно.

Мобы могут быть двух типов:

\begin{itemize}
    \item мобы ближнего боя с оружием ближнего боя;
    \item мобы дальнего боя с оружием дальнего боя.
\end{itemize}

За создание конкретного моба отвечает паттерн «Абстрактная фабрика». Абстрактная фабрика MobAbstractFactory является родителем двух дочерних фабрик:


\begin{itemize}
    \item фабрика создания моба ближнего боя;
    \item фабрика создания моба дальнего боя.
\end{itemize}

Например: игра приняла решение создать 2 моба ближнего боя и 3 моба дальнего боя, тогда фабрика мобов дальнего боя исполнит метод \texttt{CreateMob()} 3 раза (\texttt{CreateMob()} рандомно исполняет один из методов \texttt{CreateBowMob()}, \texttt{CreateCrossbowMob()} и т.д., каждый из этих методов создает экземпляр класса Mob и “дает ему в руки” соответствующее оружие). Аналогично работает фабрика мобов ближнего боя. 


\subsection*{Предметы}

За создание предметов на карте (которые можно в дальнейшем будет подобрать нашему Герою) отвечает паттерн «Абстрактная фабрика». ItemAbstractFactory является родителем двух фабрик по созданию предметов:


\begin{itemize}
    \item ArmorFactory
    \item WeaponFactory
\end{itemize}

Каждая из этих фабрик создает рандомное оружие или рандомную броню, которые наш Герой может найти в сундуках, на полу в подземельях или поднять с побежденного им моба. 

Броня имеет тип один из перечисления ArmorType, у каждого типа брони разный уровень защиты, что влияет на конечный уровень защиты персонажа. 

Оружие может быть дальнобойным и ближним, каждый из подвидов оружия имеет тип один из перечисления RangedWeaponType и  MelleWeaponType соответственно. Также для оружия дальнего боя существует определенные боеприпасы, которые подходят только к соответствующему оружию.  Боеприпасы (из перечисления AmmoType) также имеют определенное количество очков атаки, которое потом повлияет на конечный наносимый урон. 

Оружие и броня могут быть выполнены из определенного материала (из перечисления MaterialType). Материал также влияет на уровень атаки или защиты конкретного предмета.


Игрок управляет игрой с помощью клавиатуры, перед началом игры пользователь может задать клавишам команды. Взаимодействие с пользователем реализовано с помощью паттерна «Команда». После нажатия определенной клавиши (к которой была привязана команда), проверяется возможность исполнения этой команды (CanExecute()) и команда исполняется (Execute()). 

Приключения нашего героя заканчиваются, когда у Героя кончается запас здоровья (Герой умирает -- Die()), в противном случае игра будет длиться бесконечно, генерируя все новые миры. 

\end{document}
